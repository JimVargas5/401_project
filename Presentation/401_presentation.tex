\documentclass{seminar}

% From Doc.
\usepackage{amssymb,amsmath}
%\usepackage{ulem}
\usepackage{semlcmss}
\usepackage{scicomp}
\usepackage{epsfig}
\usepackage{framed}
\usepackage[T1]{fontenc}
\usepackage{multirow}
\usepackage{bm}

% From me
\usepackage{multicol}
\usepackage{ dsfont }
\usepackage{hyperref}


\def\lfoot{Portland State Applied and Computational Math Group} 
\def\rfoot{Portland State University March 11, 2020}



\begin{document} 
\pagestyle{headings}
\slidepagestyle{mypagestyle}
\centerslidesfalse

\begin{slide} %%% Title
\vspace*{\fill}
\begin{center}
{\large\bf\color{Green} An Exploration of a Lightning-Fast Laplace Solver}\\
by Jim Vargas\\
under the direction of Dr. Jeffrey Ovall\\
Portland State University\\
\end{center}
\vspace*{\fill}
\end{slide} %%% Title




\begin{slide} %%% 1
{\heading{This talk is based on...}} \small \\

\begin{center}
	\includegraphics[scale=0.2]{./PNG/llogo}\\
	%\includegraphics[scale=0.5]{./PNG/paper_abstract}
	\large{\emph{Solving Laplace Problems with Corner Singularities\\ via Rational Functions}}
\end{center}

\begin{itemize}
	\item ...A paper written by Gopal and Trefethen, published in SIAM Journal on Numerical Analysis September 2019
	\item The Lightning Laplace code, based on the paper, yields accurate approximations quickly (on nice problems)
	\item \url{https://epubs.siam.org/doi/pdf/10.1137/19M125947X}
	\item \url{https://people.maths.ox.ac.uk/trefethen/lightning.html}
\end{itemize}
\end{slide} %%% 1




\begin{slide} %%% 2
{\heading{Here's the problem}} \small \\

We wish to find a (real) function $u$ over a domain $\Omega \subset \mathds{C}$ which satisfies
\begin{align*}
\Delta u(z)=0, \quad z\in \Omega &&
u(z)=h(z), \quad z\in \Gamma .
\end{align*}
In particular, we want to be able to handle a domain with sharp corners, curves etc. 

We will find $r$, an approximation of $u$ ($u\approx\mathrm{Re}[r]$).
\begin{align*}
r(z)= \sum_{j=1}^{N_1} \frac{a_j}{z-z_j} + \sum_{j=0}^{N_2} b_j (z-z_*)^j
\end{align*}
\begin{multicols}{3}
\includegraphics[scale=.45]{./PNG/Howell_1994}
\includegraphics[scale=.4]{./PNG/snowflake}
\includegraphics[scale=.4]{./PNG/randpoly1}
\end{multicols}
\end{slide} %%% 2




\begin{slide} %%% 3
{\heading{Why this problem?}} \small \\

\begin{itemize}
	\item Problems involving the Laplace operator $\Delta={\nabla}^2$ frequently appear in physical equations:
	\begin{itemize}
		\item Heat Equation
		$\alpha {\nabla}^2 u={\partial}_t u$
		\item Schrodinger Equation
		$\left[ \frac{-{\hbar}^2}{2m}{\nabla}^2 + V \right]\Psi=i \hbar \, \partial_t \Psi$
		\item Wave Equation
		$c^2 {\nabla}^2 u={\partial}_t^2 u$
		\item And more...
	\end{itemize}
	\item Functions which satisfy Laplace's Equation have very nice properties, and are called harmonic.
	
\end{itemize}
\end{slide} %%% 3




\begin{slide} %%% 4a
{\heading{Some nice properties of functions of interest}} \small \\

\begin{itemize}
	\item The real and imaginary parts of a holomorphic (and thus also an analytic) function $f=u+iv$ are harmonic;
	\item $f$ is also smooth (infinitely differentiable) in an open set $\Omega$; by extension this applies to $u$ and $v$ as well.
	\item Maximum Principle: a harmonic function on a compact domain attains a max. (and min.) on the boundary, so $||u||_{\Omega}=||u||_{\Gamma}$.
\end{itemize}
\end{slide} %%% 4a

\begin{slide} %%% 4b
{\heading{Some nice properties of functions of interest}} \small \\

\begin{itemize}
	\item The real and imaginary parts of a holomorphic (and thus also an analytic) function $f=u+iv$ are harmonic;
	\item $f$ is also smooth (infinitely differentiable) in an open set $\Omega$; by extension this applies to $u$ and $v$ as well.
	\item Maximum Principle: a harmonic function on a compact domain attains a max. (and min.) on the boundary, so $||u||_{\Omega}=||u||_{\Gamma}$.\\
\end{itemize}

If given a harmonic function $u$, we can construct a holomorphic function by finding the harmonic conjugate of $u$ (say, $v$), and then $u+i v$ will work.

The theory will deal with holomorphic/analytic functions, and the results will trickle down to our problem.

By the Max. Principle, if $r$ approximates $f=u+iv$, and $u$ behaves like $h$ on a domain boundary, then
\begin{align*}
||u-\mathrm{Re}[r]||_{\Omega} = ||u-\mathrm{Re}[r]||_{\Gamma} = ||h-\mathrm{Re}[r]||_{\Gamma}
\end{align*} 
\end{slide} %%% 4b




\begin{slide} %%% 5
{\heading{Back to the problem}} \small \\
\begin{align*}
r(z) = \underbrace{\sum_{j=1}^{N_1} \frac{a_j}{z-z_j}}_\text{"Newman"} + \underbrace{\sum_{j=0}^{N_2} b_j (z-z_*)^j}_\text{"Runge"}
\end{align*}

\begin{itemize}
	\item Using the scheme outlined in the paper, we can have root exponentially good approximations for $u$. The general task is finding coefficients $a_j$, $b_j$ such that
	\begin{align*}
	||f-r_n||_\Omega = O(e^{-C\sqrt{n}})
	\end{align*}
	\item The theorems in the paper are based on interpolation, showing existence.
	\item In the MATLAB code, the problem is solved via a least squares approach using QR factorization, and amounts to the following:
\end{itemize}
\begin{align*}
\min_{\substack{\{a_1 \ldots ,a_{N_1}\} \\ \{b_1 \ldots ,b_{N_2}\}}} \sum_{j=0}^{M} {| \mathrm{Re} [r(y_j)]-h(y_j) |}^2.
\end{align*}
\end{slide} %%% 5




\begin{slide} %%% 6a
{\heading{Describing $r$ }} \small \\
\begin{align*}
r(z)= \sum_{j=1}^{N_1} \frac{a_j}{z-z_j} + \sum_{j=0}^{N_2} b_j (z-z_*)^j
\end{align*}
\textbf{Newman Part:} built to handle corners.

\begin{multicols}{2}
\begin{itemize}
	\item The terms $z_j$ are poles, exponentially clustered near a corner on the exterior of $\Omega$ (works for spacing scaled at least $O(n^{-1/2})$).
	\item "Rational functions are more powerful than polynomials for approximating functions near singularities..."\footnote{Lloyd N. Trefethen. 2013. \emph{Approximation theory and approximation practice}, Society for Industrial and Applied Mathematics.}
\end{itemize}
\includegraphics[scale=4]{./PNG/corner_nodes_illust}
\end{multicols}
\end{slide} %%% 6a

\begin{slide} %%% 6b
{\heading{Describing $r$ }} \small \\
\begin{align*}
r(z)= \sum_{j=1}^{N_1} \frac{a_j}{z-z_j} + \sum_{j=0}^{N_2} b_j (z-z_*)^j
\end{align*}
\textbf{The Runge part:} built to handle the interior.
\begin{multicols}{2}
\begin{itemize}
	\item The term $z_*$ is an expansion point, near the middle of $\Omega$.
	\item Polynomials can approximate root exponentially well on a nice domain (going back to Runge).
\end{itemize}
\includegraphics[scale=4]{./PNG/corner_nodes_illust}
\end{multicols}
\end{slide} %%% 6b




\begin{slide} %%% 7a
{\heading{The function $\mathrm{Re}[r]$ is harmonic }} \small \\
\begin{align*}
r(z)= \sum_{j=1}^{N_1} \frac{a_j}{z-z_j} + \sum_{j=0}^{N_2} b_j (z-z_*)^j
\end{align*}
To prove $\mathrm{Re}[r]$ is harmonic, consider $f(z)=1/z$ and $g(z)=z^k$.
The function $f$ can be decomposed as $f=u+iv$, where $z=(x,y)$ and
\begin{align*}
u(x,y)=\frac{x}{x^2+y^2} &&
v(x,y)=\frac{-y}{x^2+y^2}.
\end{align*}
Taking derivatives will show that $u$ and $v$ satisfy the Cauchy-Riemann equations, ${\partial}_x u={\partial}_y v$, ${\partial}_y u=-{\partial}_x v$, meaning $f$ is holomorphic.
\end{slide} %%% 7a

\begin{slide} %%% 7b
{\heading{The function $\mathrm{Re}[r]$ is harmonic }} \small \\
\begin{align*}
r(z)= \sum_{j=1}^{N_1} \frac{a_j}{z-z_j} + \sum_{j=0}^{N_2} b_j (z-z_*)^j
\end{align*}
To prove $\mathrm{Re}[r]$ is harmonic, consider $f(z)=1/z$ and $g(z)=z^k$.
The function $f$ can be decomposed as $f=u+iv$, where $z=(x,y)$ and
\begin{align*}
u(x,y)=\frac{x}{x^2+y^2} &&
v(x,y)=\frac{-y}{x^2+y^2}.
\end{align*}
Taking derivatives will show that $u$ and $v$ satisfy the Cauchy-Riemann equations, ${\partial}_x u={\partial}_y v$, ${\partial}_y u=-{\partial}_x v$, meaning $f$ is holomorphic.\\

Writing $g$ in polar form is enough to see that $g$ is holomorphic:
\begin{align*}
g(z)=\rho^k e^{i k \theta} = \rho^k [\cos{(k\theta)} + i \sin{(k\theta)}] . \\
\end{align*}
The sum of two holomorphic functions is holomorphic, so the real part of this sum is harmonic. Applying translations and scaling as necessary give us our result.
\end{slide} %%% 7b




\begin{slide} %%% 8
{\heading{An important lemma}} \small \\

\textbf{Hermite integral formula} for rational interpolation.

Let $\Omega$ be a simply connected domain in $\mathds{C}$ bounded by a closed curve $\Gamma$, and let $f$ be analytic in that domain and extend continuously to the boundary. Let interpolation points ${\alpha}_0, \ldots ,{\alpha}_{n-1} \in \Omega$ and poles ${\beta}_0, \ldots ,{\beta}_{n-1}$ anywhere in the complex plane be given. Let $r$ be the unique type $(n-1,n)$ rational function with simple poles at $\{{\beta}_j\}$ that interpolate $f$ at $\{{\alpha}_j\}$. Then for any $z \in \Omega$,
\begin{align*}
f(z)-r(z)=\frac{1}{2 \pi i} \int_{\Gamma} \frac{\phi (z)}{\phi (t)} \frac{f(t)}{t-z}dt, \\
\phi (z) = \left. \prod_{j=0}^{n-1}(z-{\alpha}_j) \middle/ \prod_{j=0}^{n-1}(z-{\beta}_j). \right.
\end{align*}
\end{slide} %%% 8




\begin{slide} %%% 9
{\heading{First Theorem}} \small \\

Let $f$ be a bounded analytic function in the slit disk $A_\pi$ that satisfies $f(z)=O(|z|^\delta)$ as $z \to 0$ for some $\delta > 0$, and let $\theta \in (0,\pi /2)$ be fixed. Then for some $0< \rho < 1$ depending on $\theta$ but not on $f$, there exist type $(n-1,n)$ rational functions $\{r_n\}$, $1 \leq n < \infty$, such that
	\begin{align*}
	||f-r_n||_\Omega = O(e^{-C \sqrt{n}})
	\end{align*}
as $n \to \infty $ for some $C>0$, where $\Omega = \rho A_\theta$. Moreover, each $r_n$ can be taken to have simple poles only at
	\begin{align*}
	\beta_j = -e^{-\sigma j/\sqrt{n}}, \enspace 0\leq j \leq n-1,
	\end{align*}
where $\sigma >0$ is arbitrary.
\end{slide} %%% 9




\begin{slide} %%% 10
\begin{center}
\includegraphics[scale=0.8]{./PNG/A_theta_illust}
\end{center}

\begin{align*}
A_\theta=\{z \in \mathds{C} : |z|<1,\enspace |\mathrm{arg}(z)|<\theta \} &&
\Omega = \rho A_\theta
\end{align*}
\end{slide} %%% 10




\begin{slide} %%% 11
{\heading{Second Theorem}} \small \\

Let $\Omega$ be a convex polygon with corners $w_1 , \ldots , w_m$, and let $f$ be an analytic function in $\Omega$ that is analytic on the interior of each side segment and can be analytically continued to a disk near each $w_k$ with a slit along the exterior bisector there. Assume $f$ satisfies $f(z)-f(w_k)=O(|z-w_k|^\delta)$ as $z \to w_k$ for each $k$ for some $\delta >0$. There exist degree $n$ rational functions $\{r_n\},\, 1 \leq n < \infty$ such that
	\begin{align*}
	||f-r_n||_\Omega=O(e^{-C\sqrt{n}})
	\end{align*}
as $n\to \infty$ for some $C>0$. Moreover, each $r_n$ can be taken to have finite poles only at points exponentially clustered along the exterior bisectors at the corners, with arbitrary clustering parameter $\sigma$, as long as the number of poles near each $w_k$ grows at least in proportion to $n$ as $n\to \infty$.
\end{slide} %%% 11




\begin{slide} %%% 12
{\heading{Second Theorem: the idea}}\footnote{Image from Gopal, A., \& Trefethen, L. N. (2019). \emph{Solving Laplace Problems with Corner Singularities via Rational Functions}. SIAM Journal on Numerical Analysis.} \small \\
\begin{center}
\includegraphics[scale=0.4]{./PNG/polygon_illust}
\end{center}
Split $f$ into $2m$ terms, a "Newman" part and a "Runge" part:
\begin{align*}
f=\sum_{k=1}^m f_k + \sum_{k=1}^m g_k.
\end{align*}
The Runge part can be handled by previously established results, and the Newman part can be handled by applying the first theorem to each corner.
\end{slide} %%% 12




\begin{slide} %%% 13
{\heading{Some extensions}} \small \\

Numerical experiments show that:
\begin{itemize}
	\item We can get root exponentially good approximations on non-convex domains;
	\item We're not limited to sectors and convex polygons, we can have curvy edges.\\
\end{itemize}

These theorems apply to an analytic function $f$, but our problem involves a harmonic $u$. 

If we assume $u$ satisfies the corner behavior needed and $\Omega$ is simply connected, then so will $v$, where we can have an $f=u+iv$.
\end{slide} %%% 13




\begin{slide} %%% 14
{\heading{The Algorithm}} \small \\
\begin{table}[h]
	\begin{tabular}{l}
		1. Define boundary $\Gamma$, corners $w_1,\ldots, w_m$, boundary function $h$, tolerance $\varepsilon$.	\\
		2. For increasing values of $n$ with $\sqrt{n}$	approximately evenly spaced; \\
		\: 2a. fix $N_1=O(mn)$ poles $1/(z-z_k)$ clustered outside the corners; \\
		\: 2b. fix $N_2+1=O(n)$ monomials $1,(z-z_*),\ldots,(z-z_*)^{N_2}$ and set $N=N_1+N_2+1$; \\
		\: 2c. choose $M\approx 3N$ sample points on a boundary, also clustered near corners; \\
		\: 2d. evaluate at sample points to obtain an $M\times N$ matrix $A$ and $M$-vector $b$; \\
		\: 2e. solve the least-squares problem $Ax\approx b$ for the coefficient vector $x$; \\
		\: 2f. exit loop if $||Ax-b||_\infty < \varepsilon$ or if $N$ is too large or the error is growing. \\
		3. Confirm accuracy by checking the error on a finer boundary mesh. \\
		4. Construct a function to evaluate $r(z)$ based on computed coefficients $x$.
	\end{tabular}
\end{table}
\end{slide} %%% 15




\begin{slide} %%% 16
{\heading{The code}} \small \\

Code is branded "Lightning Laplace." We enter:
\begin{itemize}
	\item Corners of a polygonal-ish/curvy domain in $\mathds{C}$;
	\item Boundary data in the form of a real function handle or scalar values corresponding to the edges.
\end{itemize}
Errors are computed by comparing the procedure with a finer sampling (so not a true error).\\
\begin{center}
\includegraphics[scale=4]{./PNG/corner_nodes_illust}
\end{center}
\end{slide} %%% 16




\begin{slide} %%% 17
\small
\begin{center}
\includegraphics[scale=0.65]{./PNG/rect_code}\\
Solve time: 0.269s, Epsilon: 1e-6, dim(A)=452 x 159, {\#}poles: 58\\
\includegraphics[scale=0.55]{./PNG/rect_pwbcs}
\end{center}
\end{slide} %%% 17




\begin{slide} %%% 18
\small
\begin{center}
\includegraphics[scale=0.65]{./PNG/tri_code}\\
Solve time: 0.295s. Epsilon: 1e-8, dim(A)=736 x 273, {\#}poles: 106\\
\includegraphics[scale=0.55]{./PNG/tri_ncbcs}
\end{center}
\end{slide} %%% 18




\begin{slide} %%% 19a-d
\begin{center}
\includegraphics[scale=0.7]{./PNG/circ_example}
\end{center}
\end{slide}
\begin{slide}
\begin{center}
\includegraphics[scale=0.7]{./PNG/ushape_example}
\end{center}
\end{slide}
\begin{slide}
\begin{center}
\includegraphics[scale=0.7]{./PNG/diamond_example}
\end{center}
\end{slide}
\begin{slide}
\begin{center}
\includegraphics[scale=0.7]{./PNG/jigsaw_example}
\end{center}
\end{slide} %%% 19a-d




\begin{slide} %%% 20a-d
\begin{center}
\includegraphics[scale=0.7]{./PNG/squarePacMan1}
\end{center}
\end{slide}
\begin{slide}
\begin{center}
\includegraphics[scale=0.7]{./PNG/squarePacMan2}
\end{center}
\end{slide}
\begin{slide}
\begin{center}
\includegraphics[scale=0.7]{./PNG/squarePacMan3}
\end{center}
\end{slide}
\begin{slide}
\begin{center}
\includegraphics[scale=0.7]{./PNG/squarePacMan4}
\end{center}
\end{slide} %%% 20a-d




\begin{slide} %%% 21a-d
\begin{center}
\includegraphics[scale=0.7]{./PNG/rectSlit1}
\end{center}
\end{slide}
\begin{slide}
\begin{center}
\includegraphics[scale=0.7]{./PNG/rectSlit2}
\end{center}
\end{slide}
\begin{slide}
\begin{center}
\includegraphics[scale=0.7]{./PNG/rectSlit3}
\end{center}
\end{slide} %%% 21a-d




\begin{slide} %%% 22a-b
\begin{center}
\includegraphics[scale=0.7]{./PNG/square-ception}
\end{center}
\end{slide}
\begin{slide}
\begin{center}
\includegraphics[scale=0.7]{./PNG/square-ception_reckoning}
\end{center}
\end{slide} %%% 22a-b




\begin{slide} %%% 23a-d
\begin{center}
\includegraphics[scale=0.7]{./PNG/cusp1}
\end{center}
\end{slide}
\begin{slide}
\begin{center}
\includegraphics[scale=0.7]{./PNG/cusp2}
\end{center}
\end{slide}
\begin{slide}
\begin{center}
\includegraphics[scale=0.7]{./PNG/cusp3}
\end{center}
\end{slide}
\begin{slide}
\begin{center}
\includegraphics[scale=0.7]{./PNG/cardioid}
\end{center}
\end{slide} %%% 23a-d




\begin{slide} %%% 24a
{\heading{What else...}} \small \\

Variants:
\begin{itemize}
	\item Discontinuous boundary conditions: pretty much no problem, except supremum norm convergence to zero. Instead, use a supremum norm weighted by distance to the nearest corner.
	\item Multiply connected domains: very much a problem.
	\item Poisson equation $\Delta u = f$ with boundary condition $h$: find $\Delta v = f$ with arbitrary bd. conds., then find $w=u-v$. $\Delta w =0$ in $\Omega$, and $w=h-v_h$ on $\Gamma$.
	\item Faster than root exponential convergence: not possible.
	\item Domain size, matrix size: Runge part of $r$ works less optimally far away from $z_*$; $m$ corners means operation count $O(m^3 |\log{(\varepsilon)}|^6)$.
\end{itemize}
\end{slide} %%% 24a

\begin{slide} %%% 24b
{\heading{What else...}} \small \\

\begin{itemize}
	\item Authors say Finite Element Methods can't match Lightning Laplace's simplicity and performance.
	\item Boundary Integral Equations are good when applicable, and is "the most powerful tool currently available."
	\item Proofs for various more general domains
	\item Expanding code functionality and usability 
\end{itemize}
\end{slide} %%% 24b




\begin{slide} %%% End
\vspace*{\fill}
\begin{center}
"The method is very young, and there are innumerable questions to be answered and details to be improved in further research. Since the method exploits the same mathematics that makes lightning strike trees and buildings at sharp edges, we like to think of it as a 'Lightning Laplace solver.'"
\end{center}
\vspace*{\fill}
\begin{center}
{\large\bf\color{Green} The End}
\end{center}
\vspace*{\fill}
\end{slide}





\end{document}